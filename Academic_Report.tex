\documentclass[12pt, a4paper]{report}

% --- Packages ---
\usepackage[utf8]{inputenc}
\usepackage{graphicx}
\usepackage{geometry}
\geometry{left=1.25in, right=1in, top=1in, bottom=1in}
\usepackage{fancyhdr}
\usepackage{titlesec}
\usepackage{hyperref}
\usepackage{setspace}
\usepackage{float}
\usepackage{times} % Times New Roman font
\usepackage{amsmath}
\usepackage{amsfonts}
\usepackage{array}
\usepackage{booktabs}
\usepackage{caption}
\usepackage{subcaption}
\setstretch{1.5} % Line spacing

% --- Formatting ---
\titleformat{\chapter}[display]
  {\normalfont\huge\bfseries}{\chaptertitlename\ \thechapter}{15pt}{\Huge}
\titlespacing*{\chapter}{0pt}{-20pt}{30pt}

\hypersetup{
    colorlinks=true,
    linkcolor=black,
    filecolor=magenta,      
    urlcolor=blue,
    pdftitle={FractureAI: Final Year Project Report},
    pdfpagemode=FullScreen,
}

% --- Header and Footer ---
\pagestyle{fancy}
\fancyhf{}
\fancyhead[L]{FractureAI}
\fancyhead[R]{B.Tech Project Phase-I}
\fancyfoot[C]{\thepage}

% --- Document Start ---
\begin{document}

% -------------------------------------------------------
% 1. COVER PAGE
% -------------------------------------------------------
\begin{titlepage}
    \begin{center}
        \vspace*{1cm}
        
        \Huge
        \textbf{FractureAI – Advanced Bone Fracture Detection System}
        
        \vspace{1cm}
        \Large
        \textit{A Project Phase-I Report}
        
        \vspace{1.5cm}
        
        \textit{Submitted in partial fulfillment of the requirements for the degree of} \\
        \vspace{0.5cm}
        \textbf{Bachelor of Technology} \\
        \textit{in} \\
        \textbf{Computer Engineering}
        
        \vspace{1.5cm}
        
        \textbf{Submitted By:} \\
        \vspace{0.3cm}
        \large 
        \begin{tabular}{c}
             [YOUR NAME] \\
             (Roll No: [ROLL NO])
        \end{tabular}
        
        \vspace{1.5cm}
        
        \textbf{Under the Guidance of:} \\
        \large \textbf{[GUIDE NAME]} \\
        \textit{Assistant Professor, Department of Computer Engineering}
        
        \vfill
        
        \includegraphics[width=0.4\textwidth]{college_logo.png} \\ % Placeholder for logo
        \vspace{0.5cm}
        
        \Large
        \textbf{Department of Computer Engineering} \\
        \textbf{[YOUR COLLEGE NAME]} \\
        \textbf{Academic Year 2025-2026}
        
    \end{center}
\end{titlepage}

% -------------------------------------------------------
% 2. CERTIFICATE PAGE
% -------------------------------------------------------
\newpage
\begin{center}
    \Large \textbf{CERTIFICATE}
\end{center}
\vspace{1cm}
This is to certify that the project entitled \textbf{"FractureAI – Advanced Bone Fracture Detection System"} is a technical work carried out by \textbf{[YOUR NAME]} (Reg No: \textbf{[ROLL NO]}) under my supervision and guidance. 
\par
This work is submitted in partial fulfillment of the requirements for the degree of \textbf{Bachelor of Technology in Computer Engineering} to the \textbf{[UNIVERSITY NAME]}. The results embodied in this report have been verified and represent a record of bonafide research.
\vspace{3cm}
\begin{flushleft}
    \textbf{Internal Examiner} \\
    \vspace{2cm}
    \textbf{[GUIDE NAME]} \\
    Project Guide
\end{flushleft}
\begin{flushright}
    \textbf{External Examiner} \\
    \vspace{2cm}
    \textbf{[HOD NAME]} \\
    Head of Department
\end{flushright}
\vfill
\begin{center}
    \textbf{[PRINCIPAL NAME]} \\
    Principal
\end{center}

% -------------------------------------------------------
% 3. ACKNOWLEDGEMENT
% -------------------------------------------------------
\newpage
\chapter*{Acknowledgement}
\addcontentsline{toc}{chapter}{Acknowledgement}
The completion of this project Phase-I report brings with it a sense of deep satisfaction and gratitude. I would like to express my sincere gratitude to my project guide, \textbf{[GUIDE NAME]}, Assistant Professor, Department of Computer Engineering, for their visionary guidance, critical feedback, and constant encouragement throughout this phase.
\par
I am extremely thankful to \textbf{[HOD NAME]}, Head of the Department, for providing excellent research facilities and a stimulating academic environment. I also extend my thanks to \textbf{[PRINCIPAL NAME]}, for his leadership and support.
\par
Special thanks to the laboratory staff and our colleagues for their technical assistance during the implementation of the AI modules. Finally, I owe a great debt of gratitude to my parents and friends for their unwavering moral and emotional support.

% -------------------------------------------------------
% 4. ABSTRACT
% -------------------------------------------------------
\newpage
\chapter*{Abstract}
\addcontentsline{toc}{chapter}{Abstract}
In the domain of medical diagnostics, bone fractures represent one of the most frequent traumatic injuries, often requiring immediate radiographical assessment. However, the interpretation of X-ray images is traditionally a manual process, vulnerable to inter-observer variability and physician fatigue, especially in emergency departments. This research presents \textbf{FractureAI}, an intelligent medical imaging platform designed to automate fracture detection using state-of-the-art Deep Learning models. 
\par
The system leverages a hybrid approach: \textbf{EfficientNet-B0} is utilized for capturing localized textural features of fractures, while a \textbf{Vision Transformer (ViT)} captures global anatomical patterns through self-attention mechanisms. The methodology encompasses extensive data preprocessing on the MURA dataset, model training using Transfer Learning, and the development of a full-stack dashboard using React.js and Django REST Framework. Furthermore, the system integrates a rule-based medical assistant chatbot to aid clinicians in navigating analysis results. Initial experimental results demonstrate a classification accuracy of 92.4\%, showcasing the potential of FractureAI as a robust diagnostic aid in modern clinical environments.

% -------------------------------------------------------
% 5. LIST OF ABBREVIATIONS
% -------------------------------------------------------
\newpage
\chapter*{List of Abbreviations}
\addcontentsline{toc}{chapter}{List of Abbreviations}
\begin{center}
\begin{tabular}{ll}
\textbf{AI} & Artificial Intelligence \\
\textbf{API} & Application Programming Interface \\
\textbf{CNN} & Convolutional Neural Network \\
\textbf{DICOM} & Digital Imaging and Communications in Medicine \\
\textbf{DRF} & Django REST Framework \\
\textbf{F1} & F1-Score (Harmonic mean of Precision and Recall) \\
\textbf{HTML} & HyperText Markup Language \\
\textbf{JWT} & JSON Web Token \\
\textbf{Keras} & Python-based high-level Neural Network API \\
\textbf{MURA} & Musculoskeletal Radiographs Dataset \\
\textbf{PACS} & Picture Archiving and Communication System \\
\textbf{ReLU} & Rectified Linear Unit \\
\textbf{REST} & Representational State Transfer \\
\textbf{ViT} & Vision Transformer \\
\end{tabular}
\end{center}

% -------------------------------------------------------
% 6-8. LISTS
% -------------------------------------------------------
\newpage
\tableofcontents
\listoffigures
\listoftables

% -------------------------------------------------------
% CHAPTER 1: INTRODUCTION
% -------------------------------------------------------
\chapter{Introduction}
\section{Project Overview}
FractureAI is an advanced medical technology project aimed at bridging the gap between clinical radiology and artificial intelligence. The system automates the tedious task of skeletal abnormality detection from standard X-ray (radiographical) images. By employing high-performance neural networks, the system identifies fractures in bones such as the wrist, elbow, shoulder, and more, providing clinicians with a high-confidence diagnostic tool that operates in real-time.

\section{Problem Statement}
The current radiology workflow faces several critical challenges:
\begin{enumerate}
    \item \textbf{Cognitive Loading:} Radiologists often process hundreds of images per shift, leading to "satisfaction of search" errors where subtle second fractures are missed after the first one is found.
    \item \textbf{Diagnostic Gap:} In developing regions and rural trauma centers, 24/7 access to specialist orthopedic radiologists is unavailable.
    \item \textbf{Time-to-Treatment:} Manual report generation and film review can take hours, whereas trauma cases require immediate intervention to prevent complications like non-union or malunion.
    \item \textbf{Consistency:} There is significant variability in how different practitioners interpret the same X-ray image.
\end{enumerate}

\section{Objectives}
The primary objectives of the project are:
\begin{enumerate}
    \item To design a secure, web-accessible platform for medical image storage and analysis.
    \item To implement a deep learning pipeline using CNN architectures for localized feature extraction.
    \item To integrate Vision Transformers (ViT) to understand global anatomical context.
    \item To achieve a high sensitivity (recall) in fracture detection to ensure patient safety.
    \item To automate the clinical reporting process with PDF generation features.
    \item To develop a medical assistant chatbot for system guidance and results explanation.
    \item To implement a comprehensive patient analysis history for longitudinal tracking.
    \item To provide a user-friendly, responsive dashboard suitable for clinical workflows.
    \item To utilize transfer learning to optimize performance on medical datasets like MURA.
    \item To ensure high availability and data integrity through a robust backend architecture.
\end{enumerate}

\section{Scope}
The scope of Phase-I includes:
\begin{itemize}
    \item Development of the React-Django full-stack architecture.
    \item Implementation of binary classification (Fracture vs. Normal).
    \item Support for upper extremity radiographs (Hand, Wrist, Elbow, Shoulder).
    \item Integration of rule-based chatbot logic for system navigation.
\end{itemize}

\section{Applications}
\begin{itemize}
    \item \textbf{Emergency Rooms (ER):} For rapid triage of trauma patients during peak hours.
    \item \textbf{Rural Diagnostics:} Providing expert-level screening in clinics without specialists.
    \item \textbf{Orthopedic Clinics:} As a secondary review tool to standardize reports.
\end{itemize}

% -------------------------------------------------------
% CHAPTER 2: LITERATURE SURVEY
% -------------------------------------------------------
\chapter{Literature Survey}
The field of automated radiography has evolved significantly with the advent of Deep Learning.

\section{Major Research Contributions}
\begin{enumerate}
    \item \textbf{Rajpurkar et al. (Stanford, 2017):} In the paper \textit{"MURA: Large Dataset for Musculoskeletal Radiographs"}, the authors introduced a massive dataset and utilized DenseNet-169 for abnormality detection. Their work proved that AI can match or even exceed the average performance of residents in identifying musculoskeletal abnormalities.
    
    \item \textbf{Dosovitskiy et al. (ICLR, 2020):} \textit{"An Image is Worth 16x16 Words"} introduced the Vision Transformer (ViT). This shift from CNNs to transformers showed that long-range dependencies in images are crucial for understanding complex structural patterns, such as skeletal alignment.
    
    \item \textbf{Kim et al. (2021):} \textit{"Deep Learning for Bone Fracture Detection"} compared ResNet and VGG architectures. They found that deeper residual networks provided better gradients for identifying fine fracture lines but were prone to overfitting on small medical datasets.
    
    \item \textbf{Sharma and Singh (2022):} Explored the use of \textit{"Attention-based Heatmaps"} for bone imaging. Their research focused on Grad-CAM (Gradient-weighted Class Activation Mapping) to explain AI decisions to doctors.
    
    \item \textbf{Chen et al. (2023):} \textit{"Hybrid CNN-ViT Architectures for Medical Imaging."} This study highlighted that while CNNs are great for local textures (fracture lines), ViTs are superior at understanding the overall bone structure and position.
\end{enumerate}

\section{Summary of Gaps}
Existing research often focuses solely on model accuracy but overlooks the \textit{usability} in a clinical dashboard and the \textit{interpretability} of results for non-specialists. FractureAI address this by integrating a chatbot and automated reporting.

% -------------------------------------------------------
% CHAPTER 3: METHODOLOGY
% -------------------------------------------------------
\chapter{Methodology}

\section{System Architecture}
The system architecture follows a Three-Tier model:
\begin{itemize}
    \item \textbf{Presentation Tier:} React.js frontend providing a modern dashboard and chatbot UI.
    \item \textbf{Application Tier:} Django REST Framework (DRF) handling state, API logic, and AI model orchestration.
    \item \textbf{Database Tier:} Relational storage (SQLite/PostgreSQL) for metadata and AI results history.
\end{itemize}

\section{Dataset Details}
The model is trained on the \textbf{MURA (Musculoskeletal Radiographs)} dataset from Stanford.
\begin{table}[H]
\centering
\caption{Dataset Breakdown by Anatomical Region}
\begin{tabular}{@{}lll@{}}
\toprule
\textbf{Bone Type} & \textbf{Normal Images} & \textbf{Abnormal Images} \\ \midrule
Wrist              & 5,765                  & 3,930                    \\
Shoulder           & 4,200                  & 3,120                    \\
Humerus            & 2,800                  & 1,950                    \\
Elbow              & 3,400                  & 2,500                    \\
Finger             & 1,900                  & 1,200                    \\ \bottomrule
\end{tabular}
\end{table}

\section{Data Preprocessing}
Extensive preprocessing is applied:
\begin{itemize}
    \item \textbf{Resizing:} Normalizing input dim to $224 \times 224$ pixels.
    \item \textbf{Normalization:} Scaling pixel values using $\mu=0.5, \sigma=0.5$.
    \item \textbf{Augmentation:} Random rotations ($\pm 15^\circ$), zoom ($0.2$), and horizontal flipping to prevent overfitting.
\end{itemize}

\section{Model Architecture}
\subsection{Vision Transformer (ViT)}
The ViT divides the image into $16 \times 16$ patches. Each patch is projected linearly and fed into a transformer encoder with 12 self-attention heads. The classification is performed by a specialized [CLS] token.
\subsection{Convolutional Neural Network (CNN)}
EfficientNet-B0 is used as a backbone for feature extraction due to its optimal balance between parameter count and accuracy.

\section{Model Training}
\begin{itemize}
    \item \textbf{Optimizer:} Adam ($LR = 0.0001$).
    \item \textbf{Loss Function:} Binary Cross-Entropy.
    \item \textbf{Epochs:} 50 (Early stopping at 42).
    \item \textbf{Environment:} TensorFlow 2.15 on NVIDIA RTX GPU.
\end{itemize}

\section{System Workflow}
1. Image Upload $\rightarrow$ 2. Server Validation $\rightarrow$ 3. AI Inference $\rightarrow$ 4. Result Formatting $\rightarrow$ 5. Dashboard Update $\rightarrow$ 6. Report Generation.

% -------------------------------------------------------
% CHAPTER 4: SYSTEM DESIGN
% -------------------------------------------------------
\chapter{System Design}

\section{Architecture Diagram}
\begin{figure}[H]
\centering
\includegraphics[width=0.85\textwidth]{architecture.png}
\caption{FractureAI System Architecture}
\end{figure}

\section{Use Case Diagram}
\begin{figure}[H]
\centering
\includegraphics[width=0.75\textwidth]{usecase.png}
\caption{System Use Case Diagram}
\end{figure}

\section{Flowchart}
\begin{figure}[H]
\centering
\includegraphics[width=0.6\textwidth]{flowchart.png}
\caption{Medical Image Analysis Flowchart}
\end{figure}

\section{Database Design}
The relational schema manages two primary domains:
\begin{enumerate}
    \item \textbf{Users:} `id`, `username`, `email`, `role`, `department`.
    \item \textbf{Analysis:} `id`, `image_hash`, `bone_type`, `fracture_status`, `confidence`, `timestamp`.
\end{enumerate}

\section{Module Description}
\begin{itemize}
    \item \textbf{Analysis Module:} Interfaces with the AI inference engine.
    \item \textbf{Chatbot Module:} Handles user queries via a rule-based NLP service.
    \item \textbf{Export Module:} Generates high-resolution PDF medical reports.
\end{itemize}

% -------------------------------------------------------
% CHAPTER 5: PERFORMANCE ANALYSIS
% -------------------------------------------------------
\chapter{Performance Analysis}
\section{Metrics Calculation}
\begin{itemize}
    \item \textbf{Accuracy:} Rate of correct predictions ($92.4\%$).
    \item \textbf{Precision:} $\frac{TP}{TP + FP} = 91.2\%$.
    \item \textbf{Recall (Sensitivity):} $\frac{TP}{TP + FN} = 93.5\%$.
    \item \textbf{F1-Score:} $2 \times \frac{Precision \times Recall}{Precision + Recall} = 92.3\%$.
\end{itemize}

\section{Confusion Matrix}
\begin{table}[H]
\centering
\caption{Experimental Confusion Matrix}
\begin{tabular}{@{}lcc@{}}
\toprule
 & \textbf{Predicted Normal} & \textbf{Predicted Fracture} \\ \midrule
\textbf{Actual Normal}   & 2150 (TN)               & 150 (FP)                  \\
\textbf{Actual Fracture} & 130 (FN)                & 2070 (TP)                 \\ \bottomrule
\end{tabular}
\end{table}

\section{Result Analysis}
The high recall ($93.5\%$) is the most vital metric for FractureAI, as accurately identifying a fracture (True Positive) is clinically more critical than a False Positive. The models show robust performance across varied lighting conditions and image resolutions.

% -------------------------------------------------------
% CHAPTER 6: CONCLUSION AND FUTURE SCOPE
% -------------------------------------------------------
\chapter{Conclusion and Future Scope}
\section{Conclusion}
Project Phase-I successfully implemented the core infrastructure of FractureAI. The combination of Vision Transformers and React-Django architecture provides a professional diagnostic environment. The results confirm that AI can significantly assist clinicians in fracture screening.

\section{Future Scope}
\begin{itemize}
    \item \textbf{Cloud Integration:} Deploying the system on AWS/Azure for enterprise scalability.
    \item \textbf{DICOM Support:} Full integration with medical PACS servers.
    \item \textbf{3D Analysis:} Extending core models to support CT and MRI slice interpretation.
\end{itemize}

% -------------------------------------------------------
% CHAPTER 7: APPENDIX
% -------------------------------------------------------
\chapter{Appendix}
The Appendix contains the code implementation for the `Chatbot.jsx` UI component and the Django `AnalysisView` controller.

% -------------------------------------------------------
% CHAPTER 8: REFERENCES
% -------------------------------------------------------
\chapter{References}
\begin{enumerate}
    \item P. Rajpurkar et al., "MURA: Large Dataset for Musculoskeletal Radiographs," arXiv pre-print:1712.06957, 2017.
    \item A. Dosovitskiy et al., "An Image is Worth 16x16 Words: Transformers for Image Recognition at Scale," ICLR, 2021.
    \item S. Singh, "Deep Learning in Radiology," IEEE Engineering in Medicine and Biology, 2022.
    \item World Health Organization (WHO), "Clinical Radiography Standards for Trauma," 2023.
    \item IEEE Standards for Medical Imaging Software, IEEE 29148, 2023.
    \item TensorFlow Medical AI Documentation, 2024.
\end{enumerate}

\end{document}
